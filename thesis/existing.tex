\chapter{Existing Work}
Most of the existing work in the area of fuzzers tend to focus on using code
coverage and similar metrics in order to direct the test inputs that are being
generated. We'll take a look at other metrics that are used by Fuzzers first,
before looking at systems that use Concolic Execution to do Path exploration to
assist in fuzzing and other similar systems.

We'll also review these existing systems to see the differences between these
systems and our design and implementation of the Confuzzer system. As well as
their respective advantages and disadvantages.

\section{Existing Fuzzers}
Most of the notable fuzzers that are in use nowadays that don't use symbolic
execution systems tend to use some form of evolutionary/genetic mutation to
generate new test cases. EFS is an example of such a system, first published in
2007 \cite{evolfuzzing}. Since then, a number of other systems have been
designed, including AutoFuzz a fuzzer aimed at trying to determine network
protocol information from an initial example of the interaction between Client
and Server and then through additional fuzzing to generate mutations of these
initial samples \cite{autofuzz}.

Most of these systems share a similar design in which an initial ``seed file''
is created with examples of an input to the program, and the system then runs by
mutating the inputs to generate new related inputs that might hit other parts of
the program execution. Fuzzers tend to have this design since otherwise a lot of
the test inputs would fail to gain any useful information as they get blocked by
the initial checks that a program might make looking for specific magic numbers
or file structures.

While this does limit fuzzers to be more adept at fuzzing certain classes of
program or requiring external seed files, the advancements in
evolutionary/mutation based fuzzers have allowed for a wide variety of commonly
used libraries and programs to be thoroughly fuzzed. We'll take a look at one of
the better non-commercial fuzzers to see what sorts of design decisions were
made and how it compares to a fuzzer based on symbolic execution principles.

\subsection{american fuzzy lop}
One of the most well-known fuzzer is ``american fuzzy lop'', a fairly effective
fuzzer developed using compile-time instrumentation and genetic algorithms to
generate additional test cases to check against the application \cite{afl}. Due
to some of the optimizatoins made over other fuzzers, and its use of
optimizations at the instrumentation level, it is one of the fastest binary-only
fuzzers available \cite{afldesign}. It has found bugs in over $75$ different
pieces of software and has multiple CVEs dedicated to bugs found by AFL.

In fact, while the software was explicitly designed to not use static analysis
or symbolic execution for performance reasons, it was still able to generate
input that included unseeded magic values by using the basic path exploration
techniques that are included as part of afl-fuzz \cite{aflsymbol}. The one
downside to afl-fuzz is that many of its features and optimizations do require
compile-time instrumentation to function and are thus less effective with
closed-source binaries.

However afl-fuzz still accomplishes a great deal while staying true to its
original design goals of being fast, useable and reliable for the general use
\cite{aflhistory}. As afl-fuzz is improved, so to are the optimizations and
performance hacks that can be used by other Fuzzing systems aimed at fuzzing
primarily closed-source binaries.

\section{Symbolic/Concolic Execution Systems}
\begin{table}
\begin{tabular}{| c | c | c | c | p{6cm} |}
\hline
System & Fuzzer & Source Code & Execution & Description\\
& & Required & Layer &\\\hline
Confuzzer & Yes & No & Binary & Fuzzing system using binary-level concolic execution.\\
KLEE & Yes & Yes & System & LLVM Extension to provide fuzzing through concolic
execution.\\
SAGE & Yes & No & Binary & Windows tool for fuzzing programs using concolic
execution.\\
s2e & No & No & System & Path Exploration tool to analyze binaries.\\
\hline
\end{tabular}
\caption{Comparison of Symbolic Execution Systems}
\label{table:othersys}
\end{table}

There are also a number of Symbolic Execution systems that use Symbolic
Execution to perform path exploration for the purpose of fuzzing or doing other
sorts of analysis on the program. While almost all the Symbolic Execution
systems have a significantly slower runtime than other Fuzzers (sometimes up to
1000x slowdown), they do provide a significant advantage to other fuzzers, as
they can find fewer test cases that cover more of the possible execution paths
through the program. Some notable ones are KLEE, SAGE, and s2e. We'll briefly go
over each of these to see how they compare to Confuzzer. Table
\ref{table:othersys} also provides an overview of these systems and their
features.

\subsection{KLEE}
KLEE is a symbolic execution tool that uses source code and the LLVM compiler to
generate tests that attain a high coverage throughout program execution
\cite{klee}. The general design of KLEE is using LLVM to create an Intermediate
Representation (IR) of the program that can be used to easily build up constraints
and symbolic formulas throughout the program execution, which can later be
solved to generate new test cases that explore other branches.

Unlike the other systems we talk about that don't require source code, KLEE is
able to use the source and generated IR in order to create complete constraints
for each instruction without any approximations. This allows almost perfect
taint analysis in exchange for the source code requirement allowing further
optimizations to be performed and tighter constraints when calculating new test
cases. However, even with the generated IR, there are still some classes of code
that can't be easily instrumented by KLEE, including floating point computation
and inline assembly code. KLEE performs fairly well as a Symbolic Fuzzer and is
able to handle many programs with available source-code.

\subsection{SAGE}
Moving on, we look over SAGE, a system designed at Microsoft that performs
symbolic fuzzing without the need for source code \cite{sage}. Between
all the existing Fuzzers and Symbolic Execution systems, SAGE ends up being one
of the closest in design and performance to Confuzzer, though primarily tested
as a Fuzzer targeting Windows executables and environments. SAGE starts with a
seed file in order to initially figure out the first path through the binary,
and during each iteration it generates new inputs by looking for new paths that
can be explored. In order to actually generate the necessary path information
and constraints, SAGE uses a combination of iDNA and TruScan which are tools
that allow offline analysis of the trace of a program, allowing SAGE to separate
the process of Taint Analysis from the actual execution of the program.

While this design does allow for more efficient Taint Analysis, it does suffer
some issues when dealing with maliciously written programs which violate the
assumptions of stack/heap space made by the Trace programs and might result in
incorrect values being stored in the program trace. In addition, this results in
SAGE having to effectively run each program execution twice, once to generate
the original program trace, and a second time to actually build up the
constraint and concrete value equations.

\subsection{S$^2$E}
S$^2$E is a system that provides automatic path exploration to allow other
programs and plugins to perform tests across all paths that a program might
traverse down. It provides this path exploration using a symbolic execution
system that analyzes the program and generates new inputs so that the program
travels down new paths. While it isn't primarily a Fuzzer, it would take little
effort to add the necessary selectors and analyzers to S$^2$E to create a
fuzzing system. Unlike most of the other Symbolic Execution systems we've
discussed, S$^2$E doesn't actually instrument the application being analyzed,
but rather instruments the entire system to control inputs at all levels of the
machine. While this does allow for more general control of the environment, it
does come at the cost of having to instrument an entire machine and therefore is
theoretically slower than equivalent binary instrumentation. 

Unlike the other symbolic execution systems, this system also primarily uses
Symbolic Execution, where it attempts to explore all paths simultaneously, and
then generates valid test cases once the paths have all been explored. While
this works for small programs and limited tainted inputs, this can result in a
combinatorial explosion in paths that might prevent the execution from finishing
in any reasonable timeframe. S$^2$E also has a focus on plugins/analyzers that
look for specific states or conditions being violated, rather than more general
problems in the program execution, making it difficult to get the full benefits
of a fully symbolic system.
