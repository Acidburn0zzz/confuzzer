\chapter{Introduction}
Since the start of Computer Security, the struggle between attackers and
defenders has been about finding vulnerabilities in software and systems that
can be exploited or protected to prevent unauthorized access. Unfortunately due
to the asymmetric nature of attackers and defenders, where attackers only need
to succeed at finding a single vulnerability and defenders have to find and
remove all vulnerabilities from a system, the traditional form of bug finding
has rarely succeeded at stopping motivated attackers.

Even with systems like Bug Bounties and Vulnerability Reward Programs, where bug
hunting is distributed among many security researchers with prizes of up to
several tens of thousands, the number of bugs in existing systems makes it
infeasible to find all of them and prevent attackers from exploiting
them \cite{googlevrp}. With the ever worsening gap between attackers and
defenders, the state of the art in finding bugs in existing code has moved from
manual inspection of the code to find bugs, to a more automated method where we
automatically generate inputs and test them against the binary, in the hopes
that one of the generated inputs will result in a bug being discovered. This
method of randomly testing inputs to cause programs to behave incorrectly is
called \textit{Fuzzing} and was invented as early as 1988 as part of a graduate
class \cite{fuzzingorigin}. Since those early days, the art of \textit{Fuzzing}
has advanced to a great degree, with advancements in how test cases are actually
generated.

\textit{Fuzzing} systems tend to be divided into 2 main categories. The first is
black-box fuzzing, where the fuzzing is performed independently of the program
that is being analyzed, and thus tends to result in random inputs being sent to
the binary, and results in many inputs that hit the same code path within the
binary. As can be imagined, black-box fuzzing is fairly inefficient and can't
completely test non-trivial binaries. The second main form of fuzzing is
white-box fuzzing, where the fuzzer has knowledge of the binary being run, and
can create test inputs based on this knowledge. If source code isn't available,
this is sometimes known as grey-box fuzzing since the feedback to the fuzzer is
from the machine code that makes up the program, rather than the original
higher-level code. The system developed for this thesis falls into the realm of
white-box/grey-box fuzzing since the inputs are very much determined by the
program execution though we don't require source code.

However, while most existing black-box fuzzing systems work off the idea of
using code coverage and input mutation to generate new test cases, the system
used in this thesis is based on \textit{Concolic Execution}. Concolic Execution
is a specific form of more general Symbolic Execution that was created as part
of the CUTE system \cite{cutesystem}. Symbolic Execution is a method of
evaluating a program, either at the higher-level source code level, or in our
case the assembly level, where each operation performed on the input data we are
testing is stored as a symbolic expression and whenever we hit some sort of
branch we store the branch as a place where different paths might be taken, and
then at the end we use all the constraints and symbolic expressions to generate
new paths through the program execution. The difference between ordinary
Symbolic Execution and Concolic Execution is how we handle branches. In Symbolic
Execution, whenever we reach a branch, we create a new thread that continues
exploring down each path in the branch, and eventually all the threads complete
and we have a massive series of constraints for all possible
executions. Meanwhile in Concolic Execution, we initially start with some
concrete values for the input and just explore down the path of execution that
those values end up taking, and then we try running the program again with new
values.

By exploring the program state and building up constraints for each possible
execution path, we can solve the constraints for different inputs that should
travel down different parts of the program, allowing us to thouroughly test
large swaths of the program with fewer inputs. By combining these two ideas
together, we hopefully will have a system that can more efficiently test closed
source programs and find vulnerabilities in them without having to wait until an
attacker starts actively exploiting the vulnerability.

In Chapter 2, we'll give an overview of existing systems in the space, as well
as the similarities and differences with the \textit{Confuzzer} system. Next in
Chapter 3, we'll talk about the design of the Taint Analysis and Concolic
Execution system. Chapter 4 talks about our actual implementation of the system
and some of the issues we've had during implementation. Then Chapter 5 will be
our evaluation of the system and its performance on some test programs. Chapter
6 will be about future work and improvements that can be made to the
system. Finally, Chapter 7 concludes the discussion on our system. We also
include an appendix showing the execution of \textit{Confuzzer} on one of our
larger test cases.
